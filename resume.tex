% !TEX program = xelatex
\documentclass{resume}
%\usepackage{zh_CN-Adobefonts_external} % Simplified Chinese Support using external fonts (./fonts/zh_CN-Adobe/)
%\usepackage{zh_CN-Adobefonts_internal} % Simplified Chinese Support using system fonts

% \usepackage{enumitem}
\usepackage{graphicx}



\begin{document}
\pagenumbering{gobble} % suppress displaying page number

\name{Jieke(Jack) Wu}

\basicInfo{
  % (412) 996-7548 $\bullet$
  \href{mailto:jack666@mail.ustc.edu.cn}{jack666@mail.ustc.edu.cn} $\bullet$
  \href{https://github.com/a-green-hand-jack}{GitHub} $\bullet$
  \href{https://a-green-hand-jack.github.io/}{Homepage}
}

\section{Education}
% Biotechnology,Department of Life Sciences and Medicine,University of Science and Technology of China
\datedsubsection{\textbf{\href{https://www.ustc.edu.cn/}{University of Science and Technology of China}}, Hefei, China}{08/2021 -- 06/2025 (expected)}

A third-year undergraduate student at the Department of Life Sciences and Medicine

% \datedsubsection{\textbf{\href{http://www.xgdfz.com/}{The Middle School Attached To Northwestern Polytechnical University}}, Xi'an, China}{08/2018 -- 06/2021}

\section{Experience}

  \datedsubsection{\textbf{Hierarchical transformer for genomics}}{Research Assistant}
  \datedsubsection{\href{https://zhanglab-aim.github.io/members/fzzhang}{Cedars-Sinai Medical Center, Dr.Zijun Zhang}\\\href{https://chenwydj.github.io/}{UC Berkeley, Dr.Wuyang Chen}}{03/2024 -- present}
    \begin{itemize}
      \item Investigating Concealed Information within DNA Sequences using Deep Learning.
      \item Exploring how to enhance the model’s perception of DNA information by integrating the global and local information of DNA sequences.
    \end{itemize}

  \datedsubsection{\textbf{Training-free Design of Data-centric Augmentations with Principles}}{Research Assistant}
  \datedsubsection{\href{https://chenwydj.github.io/}{UC Berkeley, Dr.Wuyang Chen}}{06/2023 -- 02/2024}
    \begin{itemize}
      \item Explored the influence of various image augmentation methods on the recognition accuracy of common deep learning networks.
      \item Explored the relationship between data covariance properties and image recognition accuracy.
      \item Summarized the evaluation of various image augmentation methods on different medical imaging datasets.
    \end{itemize}

    % \noindent\hfil$-$\hfil$-$\hfil$-$\hfil$-$\hfil$-$\hfil$-$\hfil$-$\hfil$-$\hfil$-$\hfil$-$\hfil$-$\hfil$-$\hfil$-$\hfil$-$\hfil$-$\hfil$-$\hfil$-$\hfil$-$\hfil
  \datedsubsection{\textbf{Isolation of bacteriophages targeting gut bacteria}}{Research Assistant}
  \datedsubsection{\href{http://biomed.ustc.edu.cn/2022/0505/c24993a553532/page.htm}{University of Science and Technology of China, Prof.Yi Duan}}{01/2023-- 05/2024}
  \begin{itemize}
    \item This study established an improved in vitro culture system for \textit{Akkermansia muciniphila} (Akk), simplifying and enhancing the user-friendliness compared to previous systems, while also eliminating \textit{Cutibacterium acnes} contamination.
    \item We successfully isolated and purified Akk-targeting phages from wastewater, facilitating the development of a phage library for gut microbiome research.
    \item The constructed phage library enables targeted Akk knockdown or knockout, advancing our understanding of Akk's role in gut-related diseases and providing a technological platform for future gut microbiota studies.
    % \item The potential to extend this approach to other gut microbiota signifies the broader applicability of our phage technology in microbiome research.
    \item This project was rated as an excellent school-level project that year.
  \end{itemize}

    % \noindent\hfil$-$\hfil$-$\hfil$-$\hfil$-$\hfil$-$\hfil$-$\hfil$-$\hfil$-$\hfil$-$\hfil$-$\hfil$-$\hfil$-$\hfil$-$\hfil$-$\hfil$-$\hfil$-$\hfil$-$\hfil$-$\hfil
  \datedsubsection{\textbf{\small{Biodegradable needles for transdermal delivery in biofilm-infected chronic wounds}}}{Research Assistant}
  \datedsubsection{\href{https://sz.ustc.edu.cn/rcdw_show/44.html}{Suzhou Institute for Advanced Research,Prof.Xiaorong Xu}}{11/2022 -- 09/2023}

    \begin{itemize}
      \item Proficiency in finite element simulation software (COMSOL and Abaqus) for conducting simulation tasks.
      % \begin{itemize}
      % \item Utilized Abaqus to simulate the process of needle insertion into the skin.
      % \item Utilized COMSOL to optimize material selection and the geometric shape of the needle.
      % \end{itemize}
      \item Designed a long needle for the treatment of deep-seated tissue infections.
      % \begin{itemize}
      % % \item The design of this long needle was inspired by the mouthparts structure of insects such as mosquitoes and ticks.
      % \item This long needle inherits the advantages of microneedles and compensates for their limitations in terms of depth.
      % \end{itemize}
      \item Introduced a novel injection molding method for the cost-effective and convenient production of long or microneedles with complex geometrical structures.
      \item This project was rated as an excellent school-level project that year.
      % \begin{itemize}
      % \item A heterogeneous model that allows simultaneous consideration of needle puncture strength and structural flexibility.
      % \end{itemize}
    \end{itemize}


    % \noindent\hfil$-$\hfil$-$\hfil$-$\hfil$-$\hfil$-$\hfil$-$\hfil$-$\hfil$-$\hfil$-$\hfil$-$\hfil$-$\hfil$-$\hfil$-$\hfil$-$\hfil$-$\hfil$-$\hfil$-$\hfil$-$\hfil
  \datedsubsection{\textbf{\small{Isolation and identification of cyanobacteria and cyanophages from Lake Chaohu}}}{Research Assistant}
  \datedsubsection{\href{https://czlab.ustc.edu.cn/}{Laboratory of Biochemistry \& Structural Biology, Prof.Congzhao Zhou}}{09/2022 -- 06/2023}
    \begin{itemize}
      % \item Studied fundamentals and principles of bioinformatics, with a focus on genome analysis techniques.
      \item Successfully isolated three strains of cyanobacteria from Lake Chaohu water samples.
      \item Conducted a genomic analysis of these three cyanobacteria strains, thereby determining their taxonomic classification.
      \item Isolated some cyanophages from Lake Chaohu water samples using these isolated cyanobacteria strains.
      % \item Observed the morphology of these cyanophages under an electron microscope.
      \item Thanks to this work, we received an award at the National University Life Science Competition in the same year.
    \end{itemize}

\section{Selected Awards}
  \begin{itemize}
      \item Outstanding School-Level Project: College Student Research Program \hfill 2023
      \item A Prize in the 8th National University Life Science Competition \hfill 2023
      \item Outstanding Undergraduate Scholarship  \hfill 2023, 2022, 2021
  \end{itemize}

% \section{Publications}
% \begin{itemize}
%     \item Shihong Song*, \textbf{Jiayi Weng}*, Hang Su, Dong Yan, Haosheng Zou, and Jun Zhu, ``Playing FPS Game with Environment-aware Hierarchical Reinforcement Learning,'' in \textbf{IJCAI'19 (oral)}, \href{https://trinkle23897.github.io/cv/viz2018.html}{[Page]} \href{https://www.ijcai.org/proceedings/2019/0482.pdf}{[PDF]}
%     \item \textbf{Jiayi Weng}, Tsung-Yi Ho, Weiqing Ji, Peng Liu, Mengdi Bao, and Hailong Yao,  ``URBER: Ultrafast Rule-Based Escape Routing Method for Large-Scale Sample Delivery Biochips,'' in \textbf{TCAD'18}, \href{https://trinkle23897.github.io/cv/urber.html}{[Page]} \href{https://ieeexplore.ieee.org/document/8552446}{[PDF]}
% \end{itemize}

\section{Skills}
  \textbf{Programming Languages:} \small Python, C/C++, Matlab (ranked by proficiency)

  \textbf{Tools and Frameworks:} \small Git, \LaTeX, PyTorch, HuggingFace


\end{document}
